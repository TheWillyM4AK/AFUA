\chapter{Introducción}

\section{Identificación}

\textbf{Identificación:}

\textbf{Título:} Advanced Flexible Use of Airspace 

\section{Propósito del sistema}

El motivo fundamental para plantear la implementación del \acrfull{afua} es conseguir un aumento de la capacidad en el sistema de gestión del tránsito aéreo. 

En los años previos al estallido de la pandemia, se ha comprobado que el sistema actual no va a ser capaz de absorber al aumento de la demanda, lo que provoca elevados valores de demora. Especialmente preocupante es el caso de los países europeos, donde debido a la gran cantidad de delegaciones existentes, el modelo actual está lejos de una organización óptima. 

Teniendo en cuenta que el volumen de tráfico volverá a sus valores de crecimiento anteriores una vez superados los efectos de la crisis sanitaria, se deben orientar las propuestas hacia una mejor utilización de todos los elementos implicados en las operaciones de tránsito aéreo. 

El concepto de \acrlong{afua}, permitirá mitigar los problemas derivados del uso compartido del espacio aéreo civil-militar. Esto permitirá dotar a las aeronaves de una mayor flexibilidad a la hora de maniobrar en un espacio aéreo tan congestionado como el europeo. 

Es evidente que la implementación de este sistema debe ir asociada a grandes mejoras en otras partes del sistema. La relación del AFUA con elementos como el sistema de \acrfull{cdm}, la implementación de un sistema Free Route y el \acrfull{swim} será fundamental a la hora de plantear su desarrollo. 

\section{Alcance del OCD}

Para lograr una aplicación impecable del concepto AFUA desde la planificación estratégica hasta la resolución táctica se necesita:

\begin{itemize}
    \item Reforzar el proceso de coordinación entre todos los socios del ATM.
    \item Mejorar el mecanismo de planificación y gestión de las rutas, la sectorización asociada y las reservas de espacio aéreo (ARES).
    \item Acercamiento de los horizontes temporales del nivel 2 de ASM (Pre-táctico) y del ATFCM-DCB (ATFCM (Air Traffic Flow and Capacity Management) que evoluciona hacia el Demand Capacity Balancing.
    \item Extender el proceso de colaboración en tiempo real desde la solicitud de espacio aéreo hasta la fase táctica (nivel 3).
    \item Mejorar el intercambio de información y la notificación de todos los datos relacionados con AFUA.
    \item Consistencia en el procesamiento y la distribución de los planes de vuelo, tanto para las operaciones civiles como para las militares.
    \item Evaluación de la mejora del desempeño asociado a AFUA a través de un enfoque proactivo.
    \item Interoperabilidad de los sistemas gracias a estándares comunes y procesos de gestión del cambio adecuados.
    \item Arquitectura del sistema comúnmente acordada.
\end{itemize}

\section{Visión general}

El concepto de \acrfull{afua} nos introduce a las denominadas como operaciones llevadas a cabo en función del rendimiento y está basado en la gestión de las distintas configuraciones del espacio aéreo. 

Gracias a este concepto se aportarán nuevas posibilidades que permitirán un uso más flexible y dinámico de los distintos elementos del sistema. 

En resumen, se creará un proceso sin fisuras, que esté basado en el \acrfull{cdm}, con una avanzada herramienta de gestión en tiempo real de las configuraciones del espacio aéreo. Además, permitirá el flujo continuo de información entre el conjunto de actores implicados en el sistema ATM, a través de una tecnología de primer nivel desarrollada dentro de los proyectos de \acrfull{sesar}.

\section{Documentos de referencia}

\begin{itemize}
    \item \textbf{Advanced FUA CONCEPT}. Publicado por Eurocontrol el 24/07/2015. 
    \item \textbf{Airspace Management Guidelines – The ASM HandBook – Airspace Management Handbook for the Application of the concept of the Flexible Use of Airspace}. Publicado por ERNIP el 23/10/2014. 
    \item \textbf{FUA AMC CADF Operations Manual – Network Operations (ed n.13)}. Publicado por Eurocontrol en 24/02/21. 
    \item \textbf{Eurocontrol Specification for the application of the Flexible Use of Airspace (FUA)}. Publicado por Eurocontrol el 10/01/2009. Ref: EUROCONTROL-SPEC-0112.
    \item \textbf{EUROCONTROL Specification for Airspace Management (ASM) Support System Requirements supporting the ASM processes at local and FAB level - Part II - ASM to ASM Systems Interface Requirements}. Publicado por Eurocontrol el 13/01/2020. Ref : EUROCONTROL-SPEC-179.
    \item \textbf{SUMMARY OF RESPONSES (SOR) DOCUMENT FOR THE Draft EUROCONTROL Specification for Airspace Management Support System Requirements supporting the ASM processes at local and FAB level - Part II - ASM to ASM Systems Interface Requirements}. Publicado por Eurocontrol el 27/09/2019. 
\end{itemize}

\section{Antecedentes}

Desde los años 90 la \acrfull{ceac} definió la nueva estrategia En-Route para aplicar en el espacio aéreo europeo. Fue en 1996 cuando se introdujo el concepto de \acrfull{fua} en los países pertenecientes a la CEAC para incrementar la capacidad del sistema de tráfico aéreo y así alcanzar uno de los objetivos marcados por la CEAC en su estrategia En-Route. 
 
En el año 2001 se produce la evolución del concepto FUA en el marco de la estrategia 2000+, cuando se adopta el concepto de un solo espacio aéreo.  El objetivo era acabar con la fragmentación del espacio aéreo de la Unión Europea y crear un espacio aéreo eficiente y seguro sin fronteras. A pesar de la desaparición de las fronteras terrestres dentro de los países que adoptaron el Tratado de Schengen, las fronteras en el espacio aéreo seguían existiendo. 
 
El concepto de FUA que se desarrolló tenía la intención mitigar los problemas que se derivan del uso compartido del espacio aéreo civil-militar, proporcionando una mayor flexibilidad del   espacio aéreo, particularmente en las áreas de mucho tráfico. Este sistema propone la integración del espacio aéreo civil-militar, eliminando la segregación que existía en el pasado. 
La necesidad de implementar un sistema que permita la integración y coordinación entre el espacio aéreo civil-militar se debe a varios factores.

\begin{itemize}
    \item En primer lugar, el incremento de tráfico que ha tenido lugar en las últimas décadas ha influido directamente en la gestión del mismo. Se estima que el tráfico aéreo global se duplica cada 15 años. Por tanto, es necesario ampliar la capacidad del espacio aéreo haciendo uso del espacio militar.
    \item En segundo lugar, los conflictos militares que han tenido lugar en los últimos años han incrementado considerablemente las operaciones militares, teniendo que segregar cada vez más el espacio aéreo civil-militar.
    \item Por último, el aumento de actos terroristas ha tenido como resultado un incremento de las operaciones nacionales de defensa que, a su vez, han congestionado aún más el espacio aéreo.
Por todo lo expuesto anteriormente, es necesario definir un sistema capaz de integrar y coordinar el espacio aéreo civil-militar con el fin de aumentar la capacidad disponible.
\end{itemize}