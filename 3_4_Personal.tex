\section{Personal}

El presente OCD propone un sistema en el que el factor humano tiene la mayor parte de responsabilidad en el funcionamiento del mismo. Por consiguiente, será el factor humano el encargado de supervisar y controlar el espacio aéreo. Se encargará también de controlar los sistemas, siendo el factor humano el que debe tomar la última decisión. Por último, deberá proponer iniciativas para mejorar su funcionamiento. Desempeñarán sus tareas en los diferentes subsistemas que componen este concepto:

\begin{itemize}
    \item \textbf{Network Manager:} Actúa como catalizador y facilitador para una gestión general de la red eficiente por parte de todas las partes interesadas de el ATM. Sus funciones están relacionadas con el diseño de espacio aéreo, la gestión de flujos y la asignación de códigos de los transponder.
    
    \item \textbf{Airspace Manager:} Su principal función es la de gestionar la demanda de operaciones civiles y militares locales o regionales, teniendo en cuenta diversos factores (restricciones, asignación de espacio aéreo, etc.). Elabora un plan de acción y lo comunica con el Network manager. 
    
    \item \textbf{ACC Supervisor:} Se encarga de gestionar todas las actividades de la sala de operaciones, decidiendo las posiciones de trabajo que deben ocupar los controladores de acuerdo a la demanda esperada. También se encarga de tomar decisiones relativas a la configuración de los diferentes sectores para ajustar la capacidad a la demanda. Los resultados de las simulaciones pueden implementarse por parte del \acrfull{atfcm} a través del \acrfull{cdm}.
    
    Por la parte militar, las tareas que se realizan son similares a las mencionadas anteriormente, excepto las relativas al ATFCM.
    
    Por tanto, proporcionará las configuraciones de tráfico adecuadas dados los requisitos del área que tienen bajo su responsabilidad.
    
    \item \textbf{Flow Management Position:} Su cometido es transmitir al Network Manager información precisa y actualizada sobre la planificación estratégica de las operaciones. Tiene acceso a información específica relacionada con eventos militares especiales que afectan un área determinada y propone soluciones para solventar los problemas.
    \item \textbf{Aproved Agency:} Son unidades autorizadas por los Estados para comunicar con los Airspace Managers sobre temas relacionados con la asignación y utilización del espacio aéreo. Sus responsabilidades incluyen:
    \begin{itemize}
        \item Presentación de sus necesidades de espacio aéreo al Airspace Manager y de cualquier actualización que soliciten.
        \item También deben asegurarse de que el uso del espacio aéreo esté de acuerdo con el plan de uso acordado.
    \end{itemize}

    \item \textbf{ATCO:} Dado que el presente OCD ubica el factor humano en el centro del sistema otorgándole una relevancia absoluta en las decisiones que se toman, el ATCO va a desempeñar un papel fundamental en el funcionamiento del sistema. Sus funciones serán:
    \begin{itemize}
        \item Controlar y coordinar el tráfico aéreo, incluida la resolución de conflictos entre operaciones militares y civiles.
        \item Utilizar los CDR y el espacio aéreo de acuerdo con la reasignación en tiempo real y la resolución de problemas específicos del espacio aéreo y/o situaciones de tráfico entre unidades ATS civiles y militares y/o unidades militares de control y/o controladores, según corresponda.
        \item Coordinar con las unidades militares ATS en tiempo real la condición táctica del uso del espacio aéreo y la disponibilidad de los CDR y los cambios en ellos.
        \item Redirigir el flujo de tráfico civil en el CDR recientemente reabierto.
        \item Para ampliar o combinar TRA o TSA, ATCO puede asignar, con poca antelación, algunos niveles de vuelo de un segmento de ruta ATS para uso temporal de operaciones militares.
    \end{itemize}
\end{itemize}
