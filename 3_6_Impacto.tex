\section{Resumen de impacto}
\cbstart
\subsection{Impactos operativos}

La operación conjunta de aeronaves civiles y militares en un mismo espacio aéreo dará lugar a una nueva forma de operación, en la que las aeronaves militares tendrán que ser consideradas como un usuario más. Esto requerirá la coordinación conjunta de todos los usuarios que operen en un determinado espacio aéreo. Esta coordinación adicional tendrá como resultado un impacto en la operación de todas las aeronaves que comparten un mismo espacio aéreo. Estos impactos operativos podrían clasificarse de la siguiente manera.

\subsubsection{Lado militar}

Tendrán que adoptar todos aquellos sistemas y procedimientos que les permita operar y mantener una mínima de separación con las aeronaves civiles con el objetivo de asegurar la máxima seguridad en las operaciones, especialmente a muy largo plazo cuando los espacios aéreos reservados para la operación militar empleen DMAs de tipo 3. Por ello deberán adoptar todos aquellos sistemas que se han mencionado en apartados anteriores con el fin de poder cumplir con los requisitos para operar en un espacio aéreo compartido. Esto tendrá como resultado un cierto impacto tanto en los modos de operación militar (deberán tener en cuenta la presencia de otros usuarios) así como el manejo de sistemas no empleados hasta el momento (TCAS, GPWS, ACAS, etc.). El manejo de toda la información va a ser tratada de forma conjunta mediante el Network Manager, por tanto, será necesario compartir la información necesaria relativa a las operaciones militares para que pueda ser procesada y tratada en las distintas fases de operación del sistema que se han descrito en apartados anteriores. 

En cuanto al lugar de operación, dado que este sistema propone funcionar a nivel europeo, las fronteras aéreas deberán desaparecer y loas aeronaves civiles ya no tendrán que disponer de autorizaciones específicas para poder operar en un determinado Estado, sino que la cooperación de los mismos facilitará el libre tránsito de estas aeronaves permitiendo reducir el impacto generado anteriormente a la hora de cruzar una frontera aérea. 

\subsubsection{Lado civil}

La puesta en marcha del sistema propuesto va a tener un impacto muy positivo en las operaciones de las aeronaves civiles, pues les permitirá seguir unas rutas más directas. Adicionalmente las restricciones que se tenían anteriormente y que no les permitía penetrar en un espacio aéreo concreto en un determinado horario o salvo autorización previa van a desaparecer. Por tanto, se eliminarán parte de los problemas de planificación que podrían surgir anteriormente a causa de estos espacios aéreos restringidos, peligrosos o prohibidos. Por otro lado, al igual que en el lado militar, los usuarios civiles tendrán que familiarizarse con la presencia de nuevos usuarios (los militares) y tendrán que coordinarse para poder llevar a cabo las operaciones de forma conjunta. Al igual que los militares, los operadores civiles deberán ser informados por las dependencias competentes de todos los cambios que sucedan derivados del uso compartido del espacio aéreo. Por tanto, ambas partes tendrán que saber gestionar dichos cambios de una manera efectiva. Esto también constituye otro impacto en las operaciones que anteriormente no existían, ya que se planificaban con anterioridad y no podía haber cambios a corto plazo.

\subsection{Impactos organizativos}

Como se ha comentado en anteriores secciones de este documento, son varios los cambios derivados de la introducción de los nuevos conceptos implementados por el AFUA.

En primer lugar, van a aparecer cambios en las responsabilidades de todos los agentes presentes en la operación. Como se ha señalado en el apartado correspondiente a la estructura organizacional, se actualizarán las funciones de algunos de los miembros del sistema para adecuarlas a la utilización del AFUA, y también se crearán nuevos puestos específicamente diseñados para aplicar los nuevos conceptos.

Al ser un sistema que se debe implementar a nivel operativo en un entorno en el que tienen presencia multitud de diferentes agentes, que a su vez son distintos en cada uno de los países europeos, se debe planificar con antelación suficiente la implementación de los cambios. Con el objetivo de aportar suficiente flexibilidad a los diferentes miembros del sistema, se deben fijar una serie de plazos que permitan una adaptación gradual a las nuevas necesidades.

Este proceso de adaptación en las responsabilidades asumidas por los agentes del sistema debe ir íntimamente ligado con un periodo de formación.

La introducción de nuevos conceptos operacionales y de nuevo software, hará imprescindible que los trabajadores se adapten a las nuevas circunstancias. Para ello se deben establecer con suficiente previsión una serie de programas formativos aplicados a todos los niveles dentro de las estructuras de trabajo de las organizaciones.

Se debe tener en cuenta la posible aparición de problemas derivados de la resistencia al cambio. Las nuevas técnicas y procesos pueden ser rechazados inicialmente, y será una tarea clave la concienciación de las ventajas que se obtendrán una vez implantados. No se espera que esta resistencia genere un problema a gran escala, ya que con la formación adecuada y una planificación para la implantación progresiva, se hará fácil entender las ventajas derivadas de los nuevos procesos.


\subsection{Impactos durante el desarrollo}

El desarrollo del sistema va a ser complicado ya que necesita del desarrollo de otros sistemas en paralelo para lograr su implantación, como ya se vio en el apartado (\ref{tit:sistemas}). Estos procesos en paralelo son fundamentales para que el sistema pueda funcionar ya que el AFUA no trae un gran salto tecnológico sino un cambio en la política, planificación, procedimientos y operación. Es por esta razón que su desarrollo no se puede ver de forma aislada sino como la llegada de una nueva generación del ATM.

La implementación del sistema en toda la red será un camino largo y difícil por lo que en las primeras fases se empezarán con pruebas piloto en distintas regiones del espacio aéreo. Se elegirán las regiones que más preparadas estén para la implementación del sistema y en las que el impacto de las operaciones vaya a ser menor. Es decir, el sistema no se empezará implantando en sectores muy tensionados donde a largo plazo va a ser más útil, sino en otras zonas donde hay más margen de error para la implementación y estén tecnológicamente preparadas. Habrá zonas que no estén preparadas para la implementación del AFUA y tendrán que pasar por una primera fase para adaptar sus instalaciones y procedimientos. Tendrán que pasar una serie de requisitos previamente establecidos.

Las regiones en las que se empezará a probar el sistema serán idealmente del tamaño de estados pequeños. Así todo el estado y sus actividades militares empezarían a usar el sistema a la vez. En cambio, para los estados de mayor tamaño se considera demasiado difícil implantar el sistema a la vez en todo el territorio.

Los cambios normativos necesarios se llevarán a cabo mediante regulación de la Unión Europea. Esta será de obligado cumplimiento y tendrán que traspasarla a su ordenamiento los países miembros. Para los países no miembros también será obligatorio incorporar la nueva normativa. Este sistema es de los que más impacto va a generar en la política y las relaciones internacionales ya que incide en la soberanía de los países y las relaciones de aquellos que comparten frontera. Se espera que el impacto sea positivo y aumente la cooperación entre estados, pero también relaciones problemáticas pueden dificultar la implementación del AFUA.

\cbend