\section{Resumen de impacto}

\subsection{Impactos operativos}

\subsection{Impactos organizativos}

\subsection{Impactos durante el desarrollo}

El desarrollo del sistema va a ser complicado ya que necesita del desarrollo de otros sistemas en paralelo para lograr su implantación, como ya se vio en el apartado (\ref{tit:sistemas}). Estos procesos en paralelo son fundamentales para que el sistema pueda funcionar ya que el AFUA no trae un gran salto tecnológico sino un cambio en la política, planificación, procedimientos y operación. Es por esta razón que su desarrollo no se puede ver de forma aislada sino como la llegada de una nueva generación del ATM.

La implementación del sistema en toda la red será un camino largo y difícil por lo que en las primeras fases se empezarán con pruebas piloto en distintas regiones del espacio aéreo. Se elegirán las regiones que más preparadas estén para la implementación del sistema y en las que el impacto de las operaciones vaya a ser menor. Es decir, el sistema no se empezará implantando en sectores muy tensionados donde a largo plazo va a ser más útil, sino en otras zonas donde hay más margen de error para la implementación y estén tecnológicamente preparadas. Habrá zonas que no estén preparadas para la implementación del AFUA y tendrán que pasar por una primera fase para adaptar sus instalaciones y procedimientos. Tendrán que pasar una serie de requisitos previamente establecidos.

Las regiones en las que se empezará a probar el sistema serán idealmente del tamaño de estados pequeños. Así todo el estado y sus actividades militares empezarían a usar el sistema a la vez. En cambio, para los estados de mayor tamaño se considera demasiado difícil implantar el sistema a la vez en todo el territorio.

Los cambios normativos necesarios se llevarán a cabo mediante regulación de la Unión Europea. Esta será de obligado cumplimiento y tendrán que traspasarla a su ordenamiento los países miembros. Para los países no miembros también será obligatorio incorporar la nueva normativa. Este sistema es de los que más impacto va a generar en la política y las relaciones internacionales ya que incide en la soberanía de los países y las relaciones de aquellos que comparten frontera. Se espera que el impacto sea positivo y aumente la cooperación entre estados, pero también relaciones problemáticas pueden dificultar la implementación del AFUA.
