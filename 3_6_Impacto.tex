\section{Resumen de impacto}
\cbstart
\subsection{Impactos operativos}

La operación conjunta de aeronaves civiles y militares en un mismo espacio aéreo dará lugar a una nueva forma de operación, en la que las aeronaves militares tendrán que ser consideradas como un usuario más. Esto requerirá la coordinación conjunta de todos los usuarios que operen en un determinado espacio aéreo. Esta coordinación adicional tendrá como resultado un impacto en la operación de todas las aeronaves que comparten un mismo espacio aéreo. Estos impactos operativos podrían clasificarse de la siguiente manera.

\subsubsection{Lado militar}

Tendrán que adoptar todos aquellos sistemas y procedimientos que les permita operar y mantener una mínima de separación con las aeronaves civiles con el objetivo de asegurar la máxima seguridad en las operaciones, especialmente a muy largo plazo cuando los espacios aéreos reservados para la operación militar empleen DMAs de tipo 3. Por ello deberán adoptar todos aquellos sistemas que se han mencionado en apartados anteriores con el fin de poder cumplir con los requisitos para operar en un espacio aéreo compartido. Esto tendrá como resultado un cierto impacto tanto en los modos de operación militar (deberán tener en cuenta la presencia de otros usuarios) así como el manejo de sistemas no empleados hasta el momento (TCAS, GPWS, ACAS, etc.). El manejo de toda la información va a ser tratada de forma conjunta mediante el Network Manager, por tanto, será necesario compartir la información necesaria relativa a las operaciones militares para que pueda ser procesada y tratada en las distintas fases de operación del sistema que se han descrito en apartados anteriores. 

En cuanto al lugar de operación, dado que este sistema propone funcionar a nivel europeo, las fronteras aéreas deberán desaparecer y loas aeronaves civiles ya no tendrán que disponer de autorizaciones específicas para poder operar en un determinado Estado, sino que la cooperación de los mismos facilitará el libre tránsito de estas aeronaves permitiendo reducir el impacto generado anteriormente a la hora de cruzar una frontera aérea. 

\subsubsection{Lado civil}

La puesta en marcha del sistema propuesto va a tener un impacto muy positivo en las operaciones de las aeronaves civiles, pues les permitirá seguir unas rutas más directas. Adicionalmente las restricciones que se tenían anteriormente y que no les permitía penetrar en un espacio aéreo concreto en un determinado horario o salvo autorización previa van a desaparecer. Por tanto, se eliminarán parte de los problemas de planificación que podrían surgir anteriormente a causa de estos espacios aéreos restringidos, peligrosos o prohibidos. Por otro lado, al igual que en el lado militar, los usuarios civiles tendrán que familiarizarse con la presencia de nuevos usuarios (los militares) y tendrán que coordinarse para poder llevar a cabo las operaciones de forma conjunta. Al igual que los militares, los operadores civiles deberán ser informados por las dependencias competentes de todos los cambios que sucedan derivados del uso compartido del espacio aéreo. Por tanto, ambas partes tendrán que saber gestionar dichos cambios de una manera efectiva. Esto también constituye otro impacto en las operaciones que anteriormente no existían, ya que se planificaban con anterioridad y no podía haber cambios a corto plazo.

\subsection{Impactos organizativos}

Como se ha comentado en anteriores secciones de este documento, son varios los cambios derivados de la introducción de los nuevos conceptos implementados por el AFUA.

En primer lugar, van a aparecer cambios en las responsabilidades de todos los agentes presentes en la operación. Como se ha señalado en el apartado correspondiente a la estructura organizacional, se actualizarán las funciones de algunos de los miembros del sistema para adecuarlas a la utilización del AFUA, y también se crearán nuevos puestos específicamente diseñados para aplicar los nuevos conceptos.

Al ser un sistema que se debe implementar a nivel operativo en un entorno en el que tienen presencia multitud de diferentes agentes, que a su vez son distintos en cada uno de los países europeos, se debe planificar con antelación suficiente la implementación de los cambios. Con el objetivo de aportar suficiente flexibilidad a los diferentes miembros del sistema, se deben fijar una serie de plazos que permitan una adaptación gradual a las nuevas necesidades.

Este proceso de adaptación en las responsabilidades asumidas por los agentes del sistema debe ir íntimamente ligado con un periodo de formación.

La introducción de nuevos conceptos operacionales y de nuevo software, hará imprescindible que los trabajadores se adapten a las nuevas circunstancias. Para ello se deben establecer con suficiente previsión una serie de programas formativos aplicados a todos los niveles dentro de las estructuras de trabajo de las organizaciones.

Se debe tener en cuenta la posible aparición de problemas derivados de la resistencia al cambio. Las nuevas técnicas y procesos pueden ser rechazados inicialmente, y será una tarea clave la concienciación de las ventajas que se obtendrán una vez implantados. No se espera que esta resistencia genere un problema a gran escala, ya que con la formación adecuada y una planificación para la implantación progresiva, se hará fácil entender las ventajas derivadas de los nuevos procesos.


\subsection{Impactos durante el desarrollo}

El desarrollo del sistema va a ser complicado ya que necesita del desarrollo de otros sistemas en paralelo para lograr su implantación, como ya se vio en el apartado (\ref{tit:sistemas}). Estos procesos en paralelo son fundamentales para que el sistema pueda funcionar ya que el AFUA no trae un gran salto tecnológico sino un cambio en la política, planificación, procedimientos y operación. Es por esta razón que su desarrollo no se puede ver de forma aislada sino como la llegada de una nueva generación del ATM.

La implementación del sistema en toda la red será un camino largo y difícil por lo que en las primeras fases se empezarán con pruebas piloto en distintas regiones del espacio aéreo. Se elegirán las regiones que más preparadas estén para la implementación del sistema y en las que el impacto de las operaciones vaya a ser menor. Es decir, el sistema no se empezará implantando en sectores muy tensionados donde a largo plazo va a ser más útil, sino en otras zonas donde hay más margen de error para la implementación y estén tecnológicamente preparadas. Habrá zonas que no estén preparadas para la implementación del AFUA y tendrán que pasar por una primera fase para adaptar sus instalaciones y procedimientos. Tendrán que pasar una serie de requisitos previamente establecidos.

Las regiones en las que se empezará a probar el sistema serán idealmente del tamaño de estados pequeños. Así todo el estado y sus actividades militares empezarían a usar el sistema a la vez. En cambio, para los estados de mayor tamaño se considera demasiado difícil implantar el sistema a la vez en todo el territorio.

Los cambios normativos necesarios se llevarán a cabo mediante regulación de la Unión Europea. Esta será de obligado cumplimiento y tendrán que traspasarla a su ordenamiento los países miembros. Para los países no miembros también será obligatorio incorporar la nueva normativa. Este sistema es de los que más impacto va a generar en la política y las relaciones internacionales ya que incide en la soberanía de los países y las relaciones de aquellos que comparten frontera. Se espera que el impacto sea positivo y aumente la cooperación entre estados, pero también relaciones problemáticas pueden dificultar la implementación del AFUA.

\subsection{Impactos en otros proyectos}

La puesta en funcionamiento del sistema que proponemos en el presente OCD, tendrá una serie de repercusiones en otros sistemas que se han propuesto en la asignatura. A continuación, se detallará el impacto en cada uno de ellos.

\subsubsection{A-CDM (Airport- Collaborative Decision Making)}

Este sistema permite mejorar la eficiencia y resiliencia de las operaciones aeroportuarias, optimizando el uso de recursos disponibles y mejorando la predictibilidad del tráfico aéreo. Esto se consigue mediante la colaboración conjunta de todos los operadores de los aeropuertos, NM y ATFCM. El intercambio de información transparente entre estos actores permitirá ofrecer unos tiempos de operación más precisos, especialmente los relacionados con la hora de salida de las aeronaves desde los aeropuertos. Obtener unos tiempos de salida más precisos tendrá un impacto positivo en la planificación sectorial (en-route), ya que unas operaciones más puntuales permitirán planificar mejor la reserva de determinados espacios aéreos. 

El AFUA, por su parte, necesitará hacer uso o disponer de la información que proporciona este tipo de sistemas ya que ayudará a la previsión de tráfico presente en un determinado sector y posibilitará hacer una sectorización más óptima y precisa que tendrá como consecuencia una segregación del espacio aéreo más reducida. Adicionalmente, una mayor previsión del tráfico en-ruta (derivada de la puntualidad aeroportuaria), reducirá las posibles modificaciones que pudieran tener lugar en la fase táctica, por lo que también reducirá la carga de trabajo que se deriva de las modificaciones en la planificación de última hora.

\subsubsection{E-AMAN (Extended Arrival Manager)}

Este sistema permite secuenciar el tráfico de llegada a un determinado aeropuerto con antelación. Los controladores en los sectores aguas arriba, que pueden estar en un centro de control diferente o incluso en un bloque de espacio aéreo funcional (FAB) diferente, obtienen avisos del sistema para proporcionar una secuencia previa de la aeronave. Mediante esos avisos, los controladores instruyen a los pilotos para que ajusten la velocidad de la aeronave a lo largo del descenso o incluso antes del inicio del descenso, reduciendo así la necesidad de espera y disminuyendo el consumo de combustible. 

Para proporcionar dichos avisos utiliza información proveniente de diversas fuentes: plan de vuelo, información radar, información meteorológica, información sobre un determinado espacio aéreo. 
El sistema que proponemos tendrá un cierto impacto en el funcionamiento de este sistema debido a la introducción de nuevos usuarios que pueden afectar la secuenciación propuesta por el E-AMAN. Por ejemplo, la reserva de un determinado espacio aéreo para ejercicios militares que tengan lugar en un determinado entorno aeroportuario, puede afectar a la secuenciación propuesta por el E-AMAN. Por otro lado, la utilización futura del espacio aéreo por aeronaves no tripuladas también puede afectar al correcto funcionamiento y secuenciación propuesta por E-AMAN. Por tanto, será necesario que este sistema tenga en consideración operaciones en el espacio aéreo de usuarios militares, RPAS o aviación general.

Por último, cabe mencionar que la implantación del AFUA tendrá como resultado una mayor carga de trabajo para los controladores ya que los nuevos usuarios necesitarán su apoyo para efectuar sus operaciones de forma segura. Dado que el E-AMAN permitirá una secuenciación automática de los flujos de tráfico, contribuirá a disminuir la carga de trabajo de los controladores que será beneficiosa para el funcionamiento del AFUA.

\subsubsection{Gestión de conflictos. Conflict Detection, Resolution and Monitoring}

El AFUA propone la coordinación del espacio aéreo con otros usuarios (aviación militar, RPAS, aviación deportiva…etc.). Esta coordinación entre los diferentes usuarios va a acarrear una serie de conflictos o problemas que estarán presentes desde la fase estratégica hasta la fase táctica. Por tanto, será necesario la presencia de un sistema capaz de resolver todos aquellos conflictos que vayan a producirse debido a la implantación del AFUA. 

El sistema de Detección, Resolución y Monitorización de conflictos tiene como función resolver todos aquellos problemas que puedan surgir en el espacio aéreo debido a la violación de las mínimas de separación entres los diferentes usuarios. Por tanto, la integración de otros usuarios en el espacio aéreo (propuesto en el AFUA) va a tener un impacto en la gestión y detección de nuevos problemas dentro del espacio aéreo compartido y va a necesitar de las soluciones que propone el sistema de Gestión de conflictos.  Esta gestión de conflictos deberá abarcar los horizontes temporales que hemos propuesto en nuestro OCD:

\begin{itemize}
    \item Nivel estratégico: se centrará en la gestión de los flujos de tráfico previstos a largo plazo.
    \item Nivel pretáctico: se centrará en la gestión de conflictos utilizando herramientas como MTCD (Medium Term Conflict Detection) 
    \item Nivel táctico: se basará en detectar y resolver los conflictos en tiempo real utilizando herramientas como el ACAS y/o STCA.
\end{itemize}

Según lo expuesto anteriormente, la interoperabilidad de los sistemas que utilicen los diferentes usuarios del espacio aéreo jugará un papel importante para poder poner en práctica este tipo de sistema que será capaz de ayudar a mejorar la seguridad de las operaciones cuando el número de usuarios presentes en un determinado espacio aéreo sea elevado.

\subsubsection{DAC (Dynamic Airspace Configuration)}

La implantación del AFUA precisará de un espacio aéreo dinámico poder poner en práctica los espacios aéreos flexibles expuestos en el capítulo 3 de este OCD. Esto permitirá reducir la segregación del espacio aéreo al permitir que los espacios reservados para determinadas operaciones sean de menor tamaño. Esta flexibilidad vendrá dada por sistemas como Dynamic Airspace Configuration, que permitirá reducir la sectorización actual para ofrecer trayectorias mas directas en función de las necesidades de los usuarios. 

La información que compartirán los diferentes usuarios del espacio aéreo, a través de los planes de vuelo de las aeronaves, serán gestionados por este sistema para ajustar las trayectorias de dichos usuarios y establecer la configuración más óptima para llevar las operaciones necesarias dentro del espacio aéreo.  Por tanto, el impacto del AFUA sobre este sistema es significativo ya que será el proveedor de las configuraciones dinámicas que se propone en el OCD.

\subsubsection{SWIM}

El SWIM se considera un elemento clave a la hora de mejorar los procesos de comunicación e intercambio de información entre los diferentes usuarios del espacio aéreo. A nivel de NM será imprescindible a la hora planificar el funcionamiento de la red, pero también dentro de los FAB y a nivel regional, para diseñar y modificar las zonas de usos flexibles de acuerdo a las necesidades. Todos los usuarios deberán tener acceso a información actualizada en las tres fases que se han propuesto en el OCD, para que las operaciones se puedan gestionar y organizar de forma centralizada, lo que favorecerá la cooperación entre los diferentes usuarios del espacio aéreo.

En el futuro cuando se implante los sistemas RPAS, el SWIM también jugará un papel clave a la hora de plantear las operaciones. Puesto que es razonable pensar que existirá una mayor congestión del espacio aéreo a bajos niveles en zonas urbanas, lo que requerirá disponer de cuanta información seamos capaces de obtener de distintas fuentes.

De nuevo, la interoperabilidad de los sistemas de comunicación entre los diferentes usuarios del espacio aéreo jugará un papel fundamental. Todos los usuarios deberán tener unos sistemas comunes capaces de compartir información útil para los demás usuarios y para los servicios ATS involucrados en proporcionar apoyo para que las operaciones se lleven a cabo de forma segura. 

\cbend