\section{Políticas y restricciones operativas}

\subsection{Principios para la implementación del AFUA}

Las políticas y principios que rigen la implementación del concepto \acrfull{afua} en el espacio aéreo europeo están desarrollados en el \citetitle{CE2150/2005} \cite{CE2150/2005}. 

La utilización flexible del espacio aéreo es un concepto de gestión del espacio aéreo, descrito por la \acrfull{oaci} y desarrollado por la \acrfull{eurocontrol}, según el cual el espacio aéreo no debe designarse como espacio aéreo puramente civil o militar, sino como un continuum en el que deben satisfacerse las necesidades de todos los usuarios en la mayor medida posible.  Eurocontrol ha recibido el mandato de asistir a la Comisión en el desarrollo de normas de aplicación de la utilización flexible del espacio aéreo. 

El concepto de \acrfull{afua} se ajustará a los siguientes principios:

\begin{itemize}
    \item La coordinación entre las autoridades civiles y militares se organizará a nivel estratégico, pretáctico y táctico mediante el establecimiento de acuerdos y procedimientos encaminados a aumentar la seguridad y la capacidad del espacio aéreo y a mejorar la eficacia y flexibilidad de las operaciones aéreas.
    
    \item Se deberá establecer y mantener la coherencia entre la gestión del espacio aéreo, la gestión de la afluencia del tránsito aéreo y las funciones de los servicios de tránsito aéreo con el fin de asegurar una eficiente planificación, distribución y utilización a todos los usuarios en los tres niveles de gestión del espacio aéreo.
    
    \item La reserva de espacio aéreo para uso exclusivo o específico de determinadas categorías de usuarios tendrá carácter temporal, se aplicará sólo durante períodos de tiempo limitados en función de la utilización real y se prescindirá de ella en cuanto cese la actividad que la haya motivado.
    
    \item Los Estados miembros cooperarán entre sí para la aplicación eficiente y coherente del concepto de utilización flexible del espacio aéreo a través de las fronteras nacionales o los límites de las regiones de información de vuelos y, en particular para atender las actividades transfronterizas. La cooperación abarcará todos los aspectos jurídicos, operativos y técnicos pertinentes.
    
    \item Las dependencias y usuarios de servicios de tránsito aéreo harán el mejor uso posible del espacio aéreo disponible.
\end{itemize}

\subsection{Restricciones operacionales}

A pesar de la desaparición de las fronteras terrestres, las fronteras en el espacio aéreos siguen estando presentes. Por esta razón, la Comisión Europea, adoptó el 10 de octubre de 2001, un conjunto de medidas con el objetivo de establecer el Cielo Único Europeo para finales del 2004. El objetivo es poner fin a la fragmentación del espacio aéreo en la Unión Europea, creando un espacio aéreo eficiente y sin fronteras.  

Las siguientes razones dificultan la puesta en marcha de este concepto, limitando su uso.  

Los estados miembros deben cooperar de forma eficiente para poder desarrollar este concepto. Esta cooperación entre las diferentes FIR debe acentuarse en las fronteras de los diferentes estados mediante la definición de una base técnica, operacional y legal común. La dificultad de esta coordinación en las fronteras entre estados es una de las mayores limitaciones del proyecto.

Por el lado militar, también existen múltiples limitaciones. En primer lugar, la falta de espacio en cabina de algunas aeronaves militares representa un gran problema a la hora de instalar sistemas más complejos y voluminosos capaces de compartir y recibir más información del entorno. Asimismo, las operaciones militares suelen estar protegidas y se evita compartir información excesiva con el exterior por motivos de seguridad (operaciones especiales, ensayos militares,etc.). Por último, cabe destacar los procesos certificación de los sistemas en el ámbito militar, que difieren, muchas veces, de los usados en la aviación civil.