\section{Políticas y restricciones operativas}

\subsection{Principios para la implementación del AFUA}

Las políticas y principios que rigen la implementación del concepto \acrfull{afua} en el espacio aéreo europeo están desarrollados en el \citetitle{CE2150/2005} \cite{CE2150/2005}. 

La utilización flexible del espacio aéreo es un concepto de gestión del espacio aéreo, descrito por la \acrfull{oaci} y desarrollado por la \acrfull{eurocontrol}, según el cual el espacio aéreo no debe designarse como espacio aéreo puramente civil o militar, sino como un continuum en el que deben satisfacerse las necesidades de todos los usuarios en la mayor medida posible.  Eurocontrol ha recibido el mandato de asistir a la Comisión en el desarrollo de normas de aplicación de la utilización flexible del espacio aéreo. 

El concepto de \acrfull{afua} se ajustará a los siguientes principios:

\begin{itemize}
    \item La coordinación entre las autoridades civiles y militares se organizará a nivel estratégico, pretáctico y táctico mediante el establecimiento de acuerdos y procedimientos encaminados a aumentar la seguridad y la capacidad del espacio aéreo y a mejorar la eficacia y flexibilidad de las operaciones aéreas.
    
    \item Se deberá establecer y mantener la coherencia entre la gestión del espacio aéreo, la gestión de la afluencia del tránsito aéreo y las funciones de los servicios de tránsito aéreo con el fin de asegurar una eficiente planificación, distribución y utilización a todos los usuarios en los tres niveles de gestión del espacio aéreo.
    
    \item La reserva de espacio aéreo para uso exclusivo o específico de determinadas categorías de usuarios tendrá carácter temporal, se aplicará sólo durante períodos de tiempo limitados en función de la utilización real y se prescindirá de ella en cuanto cese la actividad que la haya motivado.
    
    \item Los Estados miembros cooperarán entre sí para la aplicación eficiente y coherente del concepto de utilización flexible del espacio aéreo a través de las fronteras nacionales o los límites de las regiones de información de vuelos y, en particular para atender las actividades transfronterizas. La cooperación abarcará todos los aspectos jurídicos, operativos y técnicos pertinentes.
    
    \item Las dependencias y usuarios de servicios de tránsito aéreo harán el mejor uso posible del espacio aéreo disponible.
\end{itemize}

\subsection{Uso y aplicabilidad}

Como se ha mencionado anteriormente la coordinación entre las autoridades civiles y militares se organizarán en tres niveles: estratégico, práctico y táctico. Para que las operaciones puedan llevarse a cabo en estos tres niveles, es necesario definir un conjunto de políticas y normas que deberán aplicarse en cada fase para aumentar la seguridad, eficacia y flexibilidad de las operaciones aéreas.

\subsubsection{Nivel estratégico}

Los Estados miembros desempeñarán las siguientes funciones:

\begin{enumerate}
    \item Revisar con regularidad las necesidades de los usuarios, especialmente en el entorno militar. Esto contribuirá a ajustar la demanda a la capacidad del espacio aéreo y evitará los problemas en el nivel táctico.
    \item Validar las actividades que precisen de reserva o restricciones del espacio aéreo. En esta parte, la contribución del factor humano resulta imprescindible, pues será el responsable de realizar dichas validaciones.
    \item Definir estructuras temporales del espacio aéreo y procedimientos que ofrezcan opciones múltiples de reserva y rutas. Según se ha definido en apartados anteriores, será necesario hacer uso de espacio aéreos militares dinámicos. En función de la eficiencia de estos espacios se habían definido tres niveles, por tanto, la función del factor humano será determinar qué tipo de estructura es la más adecuada para el tipo de operación que va a tener lugar.
    \item Establecer criterios y procedimientos que permitan la creación y el uso de límites laterales y verticales ajustables del espacio aéreo necesario para acoger diversas variaciones de trayectorias de vuelo y cambios a corto plazo en los vuelos.
    \item Evaluar las estructuras del espacio aéreo nacional y la red de rutas con el fin de planificar estructuras y procedimientos flexibles del espacio aéreo.
    \item Determinar las condiciones específicas en las que la responsabilidad de la separación de los vuelos civiles y militares recaerá en las dependencias civiles y militares de servicios de tránsito aéreo o en las dependencias militares de control. Será el controlador el encargado de controlar que las mínimas de separación entre aeronaves civiles y militares se cumplen. 
    \item Desarrollar un uso transfronterizo del espacio aéreo con los Estados miembros cuando así lo dicten la afluencia del tránsito y las actividades de los usuarios. Para ello será necesario que todos los operadores del espacio aéreo compartan un conjunto de reglas técnicas y operacionales comunes. Sobrepasar los límites de las fronteras, ayudará a flexibilizar más el espacio aéreo, así como poder ofrecer soluciones más eficientes.
    \item Coordinar la política de gestión del espacio aéreo con la de los Estados miembros para abordar de manera conjunta la utilización del espacio aéreo a través de las fronteras nacionales y los límites de las regiones de información de vuelos.
    \item Establecer y ofrecer a los usuarios estructuras de espacio aéreo en estrecha cooperación y coordinación con los Estados miembros cuando las estructuras de espacio aéreo correspondientes tengan importantes repercusiones en el tránsito transfronterizo o en los límites de las regiones de información de vuelos con vistas a asegurar una utilización óptima del espacio aéreo a todos los usuarios de la Comunidad.
    \item Establecer con los Estados miembros un conjunto de normas comunes para la separación entre los vuelos civiles y militares en las actividades transfronterizas.
    \item Establecer mecanismos de consulta entre las personas u organismos contemplados en el apartado 3 y todas las partes y organizaciones interesadas para satisfacer debidamente las necesidades de los usuarios.
    \item Evaluar y revisar los procedimientos y el funcionamiento de las operaciones dentro de la utilización flexible del espacio aéreo.
    \item Establecer mecanismos para almacenar los datos de las solicitudes, asignación y utilización real del espacio aéreo para su posterior análisis y para la planificación de actividades.
    \item En aquellos Estados miembros en los que tanto las autoridades civiles y como las militares compartan la responsabilidad o participen en la gestión del espacio aéreo, las funciones anteriores se realizarán a través de un proceso conjunto civil-militar.
    \item Los Estados miembros identificarán a las personas y organismos responsables de asumir las funciones enumeradas en los apartados anteriores y los comunicarán a la Comisión. La Comisión mantendrá la lista de dichas personas y organismos y la publicará a fin de facilitar la cooperación entre los Estados miembros.
\end{enumerate}

\subsubsection{Nivel pretáctico}

Para este nivel se han definido las siguientes actuaciones para los Estados miembros:

\begin{enumerate}
    \item De acuerdo con lo que se ha definido en el punto 1 de la fase estratégica, los Estados miembros deben nombrar una célula de gestión encargada de asignar el espacio aéreo para cumplir con las condiciones y procedimientos definidos respecto a la revisión continua de las necesidades de los usuarios del espacio aéreo. En aquellos Estados, en los que la responsabilidad de la gestión del espacio aéreo 	recaiga sobre autoridades tanto civiles como militares, la célula que se forme debe 	adoptar un carácter conjunto civil y militar.
    
    \item Dos o más Estados miembros podrán establecer una célula conjunta de gestión del espacio aéreo. Es interesante sobre todo en zonas fronterizas, de cara a coordinar mejor las 		operaciones. Estas células tendrán como objetivo responder de una manera más 	adecuada a la demanda, centrándose en la optimización de las operaciones.

    \item Los Estados miembros deben proveer a las células de gestión de espacio aéreo con los sistemas de apoyo adecuados. Es imprescindible garantizar que serán capaces de gestionar las operaciones de asignación de espacio aéreo, así como de comunicar la disponibilidad del mismo a los distintos usuarios afectados, a otras células de gestión de espacio aéreo, a los proveedores de servicios de tránsito aéreo y a todo aquel que pueda requerir dicha información.
\end{enumerate}

\subsubsection{Nivel táctico}

Dentro del nivel táctico las actuaciones de los Estados miembros serán las siguientes:

\begin{enumerate}
    \item Deben garantizar que se crearán los procedimientos necesarios para asegurar una correcta coordinación civil y militar. También se debe asegurar una comunicación efectiva entre las dependencias de servicios de tránsito aéreo y aquellas dependencias militares de control, de manera que se permita el intercambio de información y datos acerca del espacio aéreo, para asegurar una correcta activación, desactivación o redistribución en tiempo real de los distintos espacios aéreos asignados durante el nivel pretáctico.
    
    \item Los Estados miembros deben velar porque las dependencias militares de control y las dependencias de servicios de tránsito aéreo se comuniquen mutuamente todos los cambios que puedan ocurrir durante la activación planificada del espacio aéreo. Esta comunicación debe realizarse de manera eficiente para notificar los posibles cambios a todos los usuarios afectados por la situación del espacio aéreo.
    
    \item Se debe garantizar que se introducirán los procedimientos de coordinación necesarios, así como medios de apoyo a las operaciones entre dependencias militares y dependencias de servicios de tránsito aéreo. Se debe asegurar que la gestión de la interacción de vuelos civiles y militares es absolutamente segura.
    
    \item Los Estados miembros velarán por el establecimiento de procedimientos de coordinación entre dependencias civiles y militares de servicios de tránsito aéreo, que permitan una comunicación directa de la información. Esta información está centrada en la resolución de situaciones concretas de tránsito dentro de un mismo volumen de espacio aéreo en el que estén prestando servicios tanto controladores civiles como militares.
    
    La información pertinente, se pondrá a disposición cuando sea necesario por razones de seguridad, mediante un rápido intercambio de datos de vuelo, entre los que se incluirá la posición e intención de vuelo de las aeronaves. Los destinatarios de dicha información pueden ser tanto controladores civiles, como militares.
    
    \item En el caso de operaciones transfronterizas, los Estados miembros deben garantizar que las dependencias civiles de servicios de tránsito aéreo y las dependencias militares (implicadas directamente en la provisión de servicios de tránsito aéreo o afectadas por dichas operaciones) llegan a un acuerdo de procedimientos comunes para la gestión de situaciones específicas del tránsito y para mejorar la gestión en tiempo real del espacio aéreo.
\end{enumerate}

\subsection{Restricciones operacionales}

A pesar de la desaparición de las fronteras terrestres, las fronteras en el espacio aéreos siguen estando presentes. Por esta razón, la Comisión Europea, adoptó el 10 de octubre de 2001, un conjunto de medidas con el objetivo de establecer el Cielo Único Europeo para finales del 2004. El objetivo es poner fin a la fragmentación del espacio aéreo en la Unión Europea, creando un espacio aéreo eficiente y sin fronteras. Las siguientes razones dificultan la puesta en marcha de este concepto, limitando su uso.  

Los estados miembros deben cooperar de forma eficiente para poder desarrollar este concepto. Esta cooperación entre las diferentes FIR debe acentuarse en las fronteras de los diferentes estados mediante la definición de una base técnica, operacional y legal común. La dificultad de esta coordinación en las fronteras entre estados es una de las mayores limitaciones del proyecto.

Por el lado militar, existen múltiples limitaciones, las cuales pueden agruparse en cinco bloques:

\begin{enumerate}
    \item \textbf{Limitaciones económicas:} las limitaciones presupuestarias, como consecuencia de la actividad militar sin ánimo de lucro, pueden dificultar la implementación de nuevos equipos en aviones militares a pesar de la creciente necesidad de evolucionar junto con nuevas iniciativas de ATM. Para dar solución a este problema, cada Estado, deberá subvencionar económicamente la puesta en marcha de este sistema debido a que, como se ha comentado en apartados anteriores, su implantación acarrea numerosos beneficios operacionales, medioambientales y económicos. 
    
    \item \textbf{Limitaciones operacionales:} : muchas de las operaciones militares son únicas y requieren una atención especial para llevarse a cabo. Por ejemplo, operaciones como vigilancia o patrullaje aéreo o extinción de incendios exigen la máxima prioridad y no pueden adaptarse a ninguna demora o denegación de acceso a un determinado espacio aéreo. Por ello, es necesario que el factor humano catalogue la importancia de las actividades militares que van a tener lugar y priorizar el uso del espacio aéreo en función de la importancia de las mimas.
    
    Una limitación operativa adicional viene impuesta por el artículo 3 del Convenio de Chicago, que prohíbe la operación de aeronaves estatales de un Estado sobre el territorio soberano de otro Estado sin autorización. Esta autorización puede incluir restricciones sobre cómo y dónde pueden operar dichas aeronaves. En consecuencia, las aeronaves de Estado no podrán aceptar autorizaciones ATC que modifiquen su ruta y/o altitud planificada de vuelo mientras se encuentren en el espacio aéreo soberano de otro Estado o que les hagan infringir el espacio aéreo. Por este motivo, los Estados deberán cooperar para uniformizar los modos de operación de las aeronaves militares con el fin de establecer una base común que permita cruzar fronteras libremente. 

    \item \textbf{Limitaciones técnicas:} constituye la mayor restricción para la implantación de este sistema debido a que los sistemas utilizados en las aeronaves militares difieren ligeramente de los usados en la aviación civil. Podemos encontrar diferencias en los siguientes sistemas de comunicación, navegación y vigilancia. La interoperabilidad entre los diferentes sistemas militares y civiles se discutirá en mayor detalle en los apartados siguientes. Para dar solución a este problema, cuando no pueda lograrse la interoperabilidad técnica, es decir, confiar en la capacidad de intercambio de información entre equipos o sistemas, la aplicación de procedimientos adicionales puede ser una opción alternativa para lograr el nivel de seguridad requerido y dar cabida a las aeronaves militares en el espacio aéreo en el que se admiten operaciones mixtas (civiles y militares).
    
    \item \textbf{Limitaciones en el intercambio de información:} las operaciones militares suelen estar protegidas y se evita compartir información excesiva con el exterior por motivos de seguridad (operaciones especiales, ensayos militares,etc.).  Por tanto, este aspecto restringe el intercambio de información entre los distintos participantes del espacio aéreo. 
    
    \item \textbf{Limitaciones en los procesos de certificación:} durante las últimas décadas, la evolución de la tecnología ha llevado a un aumento de los requisitos de las aeronaves cuando operan en cierto espacio aéreo (por ejemplo, RVSM, ADS-B); la evolución hacia operaciones basadas en trayectorias se sumará al aumento de los requisitos de las aeronaves. Las aeronaves militares que no pueden cumplir con los requisitos de equipamiento o certificación han buscado históricamente exenciones para operar en espacios aéreos con tales requisitos. Se espera que otorgar exenciones se vuelva más complejo en el futuro, ya que el número de requisitos y, por lo tanto, las exenciones necesarias, puede conducir a la imposibilidad (o con alta dificultad e impacto en el sistema) para compartir el mismo espacio aéreo. Por tanto, un mayor cumplimiento de las normas civiles facilitará el acceso al espacio aéreo de las aeronaves militares cuando se impongan requisitos específicos. La solución de este problema pasará por la sustitución y adopción por parte del entorno militar de los sistemas utilizados en la aviación civil.
\end{enumerate}