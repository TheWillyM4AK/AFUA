\section{Políticas y restricciones operativas}

\subsection{Políticas y normas}

Las políticas y principios que rigen la implementación del concepto \acrfull{afua} en el espacio aéreo europeo están desarrollados en el \cite{}

\subsection{Restricciones operacionales}

A pesar de la desaparición de las fronteras terrestres, las fronteras en el espacio aéreos siguen estando presentes. Por esta razón, la Comisión Europea, adoptó el 10 de octubre de 2001, un conjunto de medidas con el objetivo de establecer el Cielo Único Europeo para finales del 2004. El objetivo es poner fin a la fragmentación del espacio aéreo en la Unión Europea, creando un espacio aéreo eficiente y sin fronteras.  

Las siguientes razones dificultan la puesta en marcha de este concepto, limitando su uso.  

Los estados miembros deben cooperar de forma eficiente para poder desarrollar este concepto. Esta cooperación entre las diferentes FIR debe acentuarse en las fronteras de los diferentes estados mediante la definición de una base técnica, operacional y legal común. La dificultad de esta coordinación en las fronteras entre estados es una de las mayores limitaciones del proyecto.

Por el lado militar, también existen múltiples limitaciones. En primer lugar, la falta de espacio en cabina de algunas aeronaves militares representa un gran problema a la hora de instalar sistemas más complejos y voluminosos capaces de compartir y recibir más información del entorno. Asimismo, las operaciones militares suelen estar protegidas y se evita compartir información excesiva con el exterior por motivos de seguridad (operaciones especiales, ensayos militares,etc.). Por último, cabe destacar los procesos certificación de los sistemas en el ámbito militar, que difieren, muchas veces, de los usados en la aviación civil.