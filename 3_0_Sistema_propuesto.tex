\chapter{Sistema propuesto y esquema operativo}

Como se ha comentado en el apartado anterior, los resultados obtenidos de la implementación del \acrfull{fua} han sido dispares en el entorno europeo. Los niveles de aplicación que se han alcanzado dependen de la zona geográfica en la que nos encontremos.

Aunque sí es verdad que han aparecido consecuencias positivas de la implementación del FUA se consideran insuficientes, por lo que es necesario seguir buscando mejoras que permitan crear un sistema más sólido y con niveles de implantación regulares a lo largo del territorio europeo. Es en este contexto donde se comienza a definir el AFUA (Advanced Flexible Use of Airspace).