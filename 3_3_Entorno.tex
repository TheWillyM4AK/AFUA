\section{Entorno Operativo}

\subsection{Sistema de navegación}

\subsubsection{Perfomance Based Navigation (PBN)}

El \acrfull{pbn} es un nuevo concepto basado en el uso de los sistemas de navegación de área (RNAV). 

La navegación basada en el desempeño según OACI, hace necesario que los sistemas de desempeño requeridos en navegación (RNP) y navegación de área (RNAV) sean definidos según los términos de precisión, integridad, disposición, continuidad y funcionalidad requeridos para las operaciones propuesta en el contexto de un espacio aéreo particular, cuando está respaldado por una infraestructura de navegación apropiada (GNSS u otro tipo de infraestructura aplicable a la navegación).

El concepto PBN, incluyendo el uso de señales obtenidas de satélites para las actuaciones en ruta y aproximación, está reemplazando gradualmente al sistema tradicional de navegación apoyada en ayudas terrestres a la navegación.

El concepto de PBN está compuesto de tres partes:

\begin{enumerate}
    \item Las especificaciones de navegación. Describen los requisitos de desempeño en términos de precisión, integridad y continuidad de las operaciones propuestas en un determinado espacio aéreo. También describe cómo se van a alcanzar estos requisitos. Deben llevar asociado unos apropiados estándares de conocimiento por parte de los pilotos, así como formación adecuada.
    
    \item La infraestructura de ayudas a la navegación que se utilizarán para cumplir con las especificaciones de navegación. La disponibilidad de esta infraestructura de ayuda tiene que ser tenida en cuenta a la hora de habilitar la aplicabilidad de este tipo de navegación.
    
    \item La aplicación a la navegación está referida a las especificación y ayudas a la navegación en el contexto de un espacio aéreo con rutas ATS y procedimientos de vuelo por instrumentos.
\end{enumerate}

Los beneficios asociados a la implantación del concepto PBN son los siguientes:

\begin{itemize}
    \item Reduce la necesidad de mantener una serie de ayudas específicas para la navegación y ruta y los costes asociados del uso y mantenimiento de estas.
    \item Evita la necesidad de desarrollar ayudas específicas para las operaciones con cada nueva evolución de los sistemas de navegación, lo que supone un coste demasiado elevado.
    \item Permite usar el espacio aéreo con una mayor eficiencia.
\end{itemize}

\subsubsection{Global Navigation Satellite System (GNSS)}

El \acrfull{gnss} transmite rangos de señales que se utilizan para el posicionamiento y localización de cualquier parte del globo terrestre. Estos satélites permiten determinar coordenadas geográficas y altitud en un punto dado.

El origen de este sistema es militar. La navegación por satélite permitía alcanzar valores de precisión que no se habían conseguido obtener con anterioridad, esto permitía aumentar la precisión sobre objetivos militar, aumentando la efectividad y reduciendo daños no deseados.

En el ámbito civil, los sistemas de posicionamiento por satélite son reconocidos como un elemento clave en los sistemas de comunicación, navegación y vigilancia (CNS). 

El GNSS comprende a todos los sistemas de navegación por satélite, los que ya han sido implementados (GPS y GLONASS) y los que están en desarrollo (GALILEO). Permiten la utilización de las redes de satélite como soporte a la navegación, ofreciendo una localización precisa de las aeronaves.

Cuando se encuentre plenamente desarrollado se prevé que pueda ser utilizado, en todas las fases de la operación de una aeronave, sin requerir ayuda de cualquier otro sistema de navegación convencional

\subsubsection{Instrumental Landing System (ILS) y Microwave Landing System (MLS)}

El \acrfull{ils} es el sistema de ayuda a la aproximación y aterrizaje establecido por OACI. Este sistema permite que un avión sea guiado con presión durante la aproximación a la pista, y en algunos casos, a lo largo de la misma.

El ILS está formado por dos subsistemas independientes, uno permite proporcionar guiado lateral, y el otro proporciona guía vertical. Se trata del localizador y la senda de planeo respectivamente.

El tipo de operaciones que permite el uso del ILS son las siguientes:

\begin{itemize}
    \item \textbf{Categoría I:} Altura de decisión no inferior a 60 m (200 ft) y con una visibilidad de no menos de 800 m o con un alcance visual en la pista no inferior a 550 m.
    \item \textbf{Categoría II:} Altura de decisión inferior a 60 m (200 ft) pero no inferior a 30 m (100 ft) y con un alcance visual en la pista no inferior a 300 m.
    \item \textbf{Categoría III:}
    \begin{itemize}
        \item \textbf{Categoría IIIA:} Altura de decisión inferior a 30 m (100 ft), o sin altura de decisión y un alcance visual en la pista no inferior a 175 m.
        \item \textbf{Categoría IIIB:} Altura de decisión inferior a 15 m (50 ft), o sin altura de decisión, y un alcance visual en la pista inferior a 175 m pero no inferior a 50 m.
        \item \textbf{Categoría IIIC:} Sin altura de decisión y sin restricciones de alcance visual en la pista.
    \end{itemize}
\end{itemize}

Por su parte, el MLS, es un sistema de ayuda al aterrizaje desarrollado por el servicio militar de EEUU, cuya principal misión es paliar una de las mayores limitaciones del sistema ILS. Trata de solucionar los problemas ocasionados por la presencia de irregularidades en el terreno y las distorsiones ocasionales.

Algunas de las ventajas proporcionadas por el MLS son las siguientes:

\begin{itemize}
    \item Equipamiento más preciso.
    \item Permite múltiples curvas de aproximación, a diferencia de la rigidez de la aproximación lineal del ILS.
    \item Es más barato.
\end{itemize}

Ambos sistemas serán sustituidos en el futuro por sistemas de navegación por satélite basados en GNSS, que son mucho más precisos que ambos.

\subsection{Entorno geopolítico}

La propia existencia del nacimiento del concepto \acrfull{fua} y su evolución en el concepto \acrfull{afua} es debido al contexto geopolítico en el que nace, Europa. En la otra zona del mundo que más se ha desarrollado la aviación, EEUU, este concepto no es tan necesario ya que todo el espacio aéreo pertenece al mismo país, el cual tiene un tamaño enorme. 

En cambio en Europa se tiene un espacio aéreo muy congestionado el cual está dividido por muchos estados. Los estados de la Unión Europea tienen un tamaño menos respecto a los presentes en otras zonas del mundo. En poco territorio se concentran muchos estados y cada uno tiene su ejercito y su soberanía respecto a su cielo. Esto hace que surjan los problemas derivados de las fronteras entre cada estado y que un vuelo pase por varios países cada uno con sus áreas restringidas para uso militar.

Es lógico que el concepto FUA haya surgido y tenido un mayor desarrolla en Europa, ya que es una zona con una problemática característica que el AFUA intenta solucionar.

\subsection{Instituciones}

El concepto del AFUA no es una idea abstracta que está en el aire sino que es un concepto creado, desarrollado y sustentado por un contexto social y unas instituciones en concreto. En este apartado se van a mencionar las más importantes respecto al AFUA con una pequeña descripción de ellas.

\subsubsection{SESAR Joint Undertaking}

El \acrfull{afua} se enmarca dentro del \acrfull{sesar}, que es un proyecto colaborativo para reformar completamente el espacio aéreo europeo y su gestión del tráfico aéreo (ATM). El programa actual lo gestiona la empresa SESAR Joint Undertaking como asociación público-privada.

El proyecto SESAR lo gestiona SESAR Joint Undertaking, una empresa creada en 2007 y  responsable de la coordinación y concentración de todas las actividades de investigación y desarrollo de la Unión Europea (UE) en materia de gestión del tráfico aéreo (ATM). Iniciado en 2004, el programa SESAR es el brazo tecnológico de la iniciativa del Cielo Único Europeo de la UE para integrar los sistemas de ATM de los Estados miembros.

La empresa común SESAR se creó con Eurocontrol y la Comisión Europea como miembros fundadores. Además de los dos miembros fundadores, 15 organizaciones han firmado un acuerdo de adhesión a la empresa común SESAR. El proyecto se enmarca dentro de la Red Transeuropea de Transporte.

\subsubsection{Comisión Europea}

La Comisión Europea (CE) es la rama ejecutiva de la Unión Europea, responsable de proponer legislación, hacer cumplir las leyes de la UE y dirigir las operaciones administrativas de la unión. Los comisarios prestan juramento en el Tribunal de Justicia Europeo en la ciudad de Luxemburgo, comprometiéndose a respetar los tratados y a ser completamente independientes en el desempeño de sus funciones durante su mandato.

\subsubsection{Eurocontrol}

La Organización Europea para la Seguridad de la Navegación Aérea, comúnmente conocida como Eurocontrol, es una organización internacional que trabaja para lograr una gestión del tráfico aéreo segura y sin fisuras en toda Europa. Fundada en 1960, Eurocontrol cuenta actualmente con 41 Estados miembros y tiene su sede en Bruselas (Bélgica). También cuenta con varias sedes locales, como las actividades de I+D en Brétigny-sur-Orge (Francia), el Instituto de Formación en Navegación Aérea (IANS) en Luxemburgo y el Centro de Control de la Zona Superior de Maastricht (MUAC) en Maastricht (Países Bajos). 

Aunque Eurocontrol no es una agencia de la Unión Europea, la UE ha delegado parte de su normativa sobre el Cielo Único Europeo en Eurocontrol, lo que la convierte en la organización central de coordinación y planificación del control del tráfico aéreo para toda Europa. La propia UE es signataria de Eurocontrol y todos los Estados miembros de la UE son actualmente también miembros de Eurocontrol. La organización trabaja con las autoridades nacionales, los proveedores de servicios de navegación aérea, los usuarios del espacio aéreo civil y militar, los aeropuertos y otras organizaciones. Sus actividades abarcan todas las operaciones de los servicios de navegación aérea de puerta a puerta: gestión estratégica y táctica del flujo, formación de controladores, control regional del espacio aéreo, tecnologías y procedimientos a prueba de seguridad y recaudación de las tasas de navegación aérea.

\subsubsection{Red Transeuropea de Transporte}

La Red Transeuropea de Transporte es una red prevista de carreteras, ferrocarriles y aeropuertos en la Unión Europea. La red forma parte de un sistema más amplio de redes transeuropeas, que incluye una red de telecomunicaciones y una red energética propuesta. La Comisión Europea adoptó los primeros planes de acción sobre redes transeuropeas en 1990.

La red prevé la mejora coordinada de carreteras primarias, ferrocarriles, vías navegables interiores, aeropuertos, puertos marítimos, puertos interiores y sistemas de gestión del tráfico, proporcionando rutas integradas e intermodales de larga distancia y alta velocidad. La decisión de adoptar la red fue tomada por el Parlamento Europeo y el Consejo en julio de 1996. La UE trabaja para promover las redes mediante una combinación de liderazgo, coordinación, emisión de directrices y financiación de los aspectos de desarrollo.

